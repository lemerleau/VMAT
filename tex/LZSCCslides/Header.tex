\PassOptionsToPackage{table,dvipsnames}{xcolor}

%% ----------------------------------------------------------------------------
%% command 'usetheme' searches in directory '../LZSCCslides' for *.sty files 
\makeatletter
\def\beamer@calltheme#1#2#3{%
  \def\beamer@themelist{#2}
  \@for\beamer@themename:=\beamer@themelist\do
  {\usepackage[{#1}]{\beamer@themelocation/#3\beamer@themename}}}

\def\usefolder#1{
  \def\beamer@themelocation{#1}
}
\def\beamer@themelocation{}

\usefolder{LZSCCslides}
\usetheme[]{LUL}
%% ----------------------------------------------------------------------------

%% Site Packages %%
% Required
\usepackage[utf8]{inputenc}  	% Unterstützung für Unicode-Zeichen-Eingabe
\usepackage[T1]{fontenc}      	% Unterstützung für Europäische-Zeichen-Ausgabe
\usepackage{ae}               	% verbesserte Unterstützung für Umlaute  
\usepackage[ngerman]{babel} 	% deutsche Übersetzungen und Wortumbrüche
\usepackage{textcomp}

% Additions
\usepackage{amsmath}            % Erweiterung Mathematisch Umgebung
\usepackage{tabularx}           % Gestaltung Tabellen 
\usepackage{booktabs}           % Gestaltung Tabellen (zusaetzl. Optionen)
\usepackage{hyperref}           % Links

\usepackage{graphicx}           % Einbindung Bilder, Grafiken
\usepackage{epstopdf}           % Einbindung EPS-Bilder
\usepackage{tikz}               % Eigene Erstellung von Grafiken
\usetikzlibrary{positioning}
\usepackage{svg}                % svg-Bilder einbinden

\usepackage{xcolor}             % Verwendung Farben fuer Schrift / Elementen
\usepackage{adjustbox}          % Makros fuer Boxen / Text justierung
\usepackage{tcolorbox}          % Erstellung Boxen (Bspw. fuer Definitionen)
\usepackage{varwidth}           % Umgebung fuer Variable Textbreite (codebox)
\usepackage{setspace}			      % setstretch 

\usepackage{listings}			      % Programm-Code
\usepackage{multicol}           % mehrere Spalten
\usepackage{multirow}           % multirow fuer bessere Tabellen
 
\usepackage{pgfplots}           % HQ Funktion-Plots direkt in TeX erzeugen
\usepgfplotslibrary{dateplot}
\usepackage{xspace}             % Hilfe fuer Leerzeichen nach eigenen Makro-Defintionen

% In order to avoid BigBlueButton problems (converting errors with fonts)
\usepackage{lmodern}

\usepackage{xpatch}


%% Farben =====================================================================
% Unifarben
\definecolor{uniDarkRed}{HTML}{B02F2C}      % granat
\definecolor{uniLightRed}{HTML}{D8413E}     % karneol
\definecolor{uniLightBlue}{HTML}{8AC2D1}    % aquamarin
\definecolor{uniLightGrey}{HTML}{C9C9C9}    % grau (hell)
\definecolor{uniLightBlack}{HTML}{262A31}   % basalt
\definecolor{lightYellow}{HTML}{FFF60C}     % gelb

% Codefarben
\definecolor{pblue}{rgb}{0.13,0.13,1}
\definecolor{pgreen}{rgb}{0,0.5,0}
\definecolor{pred}{rgb}{0.9,0,0}
\definecolor{pgrey}{rgb}{0.46,0.45,0.48}


%% Einstellung Folien =========================================================

% Aufzaehlungen itemize/enumerate (Farbe)
%FORM?
% \setbeamertemplate{itemize items}{\tiny{$\blacksquare$}}
% \setbeamertemplate{itemize subitem}{$\bullet$}

\setbeamercolor{item}{fg=uniDarkRed, bg=uniLightBlack}
\setbeamercolor{subitem}{fg=uniLightRed, bg=uniLightGrey}
\setbeamercolor{subsubitem}{fg=uniLightGrey, bg=uniLightGrey}

% Fussleiste mit aktueller section
\setbeamertemplate{frame footer}{%
  \textcolor{gray}{\thesection\,|\,\textbf{\@currentlabelname}}
}

% Kapitelnummer auf den Section-Seiten, ueberschreibung THEME Einstellung
\setbeamertemplate{section page}{
  \centering
  \begin{tikzpicture}
    [remember picture, overlay]
    \draw node[anchor=south east, align=right] at (0.25,0) (A) {{
        \color{uniLightGrey!90}
        \bfseries
        \fontsize{60pt}{12pt}\selectfont
        \thesection
    }};
    % \node[align=center] at ($(A.north)+(0.0,+0.3)$) {{
    %     \color{uniLightGrey!70}\bfseries\MakeUppercase{Kapitel}
    % }};
  \end{tikzpicture}
  \hspace*{0.3cm}
  \begin{minipage}{22em}
    \raggedright
    \usebeamercolor[fg]{section title}
    \usebeamerfont{section title}
    \insertsectionhead\\[-1ex]
    \usebeamertemplate*{progress bar in section page}
    \par
    \ifx\insertsubsectionhead\@empty\else%
      \usebeamercolor[fg]{subsection title}%
      \usebeamerfont{subsection title}%
      \insertsubsectionhead
    \fi
  \end{minipage}
  \par
  %
}

% Frame title
\setbeamerfont{frametitle}{series=\mdseries}
\AtBeginDocument{\usebeamerfont{normal text}}

% Adjust Table of Contents separation
%\makeatletter
%\patchcmd{\beamer@sectionintoc}{\vskip1.5em}{\vskip0.5em}{}{}
%\makeatother


%% Tikz Settings =========================================================
% Libraries
\usetikzlibrary{calc}
\usetikzlibrary{shapes,arrows}
\usetikzlibrary{decorations.shapes}

% Block styles
\tikzset{
	block1/.style   = {rectangle, draw=uniLightGrey, thick, fill=yellow!20, align=center, rounded corners},
  block2/.style   = {rectangle, draw=black, thick, fill=uniLightBlack!20, align=center, font=\scriptsize\ttfamily},
  line/.style     = {draw, thick, -latex', shorten >=2pt},
  circle1/.style  = {circle,  draw=black, thick, align=center, font=\footnotesize\ttfamily},
  decide1/.style  = {diamond,  draw=black, thick, fill=yellow!20, align=center, font=\footnotesize\ttfamily}
}


%% User-defined Commands / Environments ==============================================
% Einrueckungen
\newcommand{\boxskip}{\hspace*{7pt}}   
\newcommand{\itemskip}{\hspace*{13pt}} % align listings mit itemize Umgebung
\newcommand{\codeskip}{\hspace*{35pt}} % align listings mit 1. Ebene Anfuehrung (mit ZEICHEN)

% Text
\newcommand{\textcode}[2]{{\ttfamily\color{#1} #2}}
\newcommand{\dbl}[0]{\textquotedbl}
\newcolumntype{C}[1]{>{\centering}p{#1}}

% Codeboxes -----------------------------------------------
%kleine Box
\newcommand{\smallcodebox}[1]{ %
  \fcolorbox{uniLightGrey!80}{white}{{\footnotesize\ttfamily #1}}\vspace*{2pt}
}

%BIg Box
%DOES NOT WORK WITH \textcode (COLOR) (?) -> use \textcolor 
%end text with \par, to avoid irregular line separation
%optional: define color for box, e.g. \simplebox[pblue]{...}
\newcommand{\simplebox}[2][black]{%
  \vspace*{2pt}
  \begin{tcolorbox}%
    [hbox, size=small,  nobeforeafter,colback=white, %
    colframe=uniLightGrey!80, boxrule = 1pt, shrink tight, %
    sharp corners, boxsep = 5pt, left=0pt]
    \color{#1}
    \begin{varwidth}{\textwidth}
      {\footnotesize\ttfamily\setstretch{1.0}
          #2
          \par
      }
    \end{varwidth}
  \end{tcolorbox}
  \vspace*{-2pt}
}


%% Code Style =================================================================
\lstdefinestyle{basic}{
  basicstyle=\ttfamily\scriptsize,
  commentstyle=\color{pgreen},
  keywordstyle=\color{pblue},
  stringstyle=\color{pred},
  moredelim=[il][\textcolor{pgrey}],
  backgroundcolor=\color{uniLightGrey!30},
  numbers=left,
  numberstyle=\tiny\color{pgrey!80},
  framexleftmargin=1em,
  tabsize=2,
  showtabs=false,
  showspaces=false,                
  showstringspaces=false,
  breaklines=true
}



%% Backup =====================================================================
%\newcommand{\backupbegin}{
%   \newcounter{framenumberappendix}
%   \setcounter{framenumberappendix}{\value{framenumber}}
%}
%\newcommand{\backupend}{
%   \addtocounter{framenumberappendix}{-\value{framenumber}}
%   \addtocounter{framenumber}{\value{framenumberappendix}} 
%}

\newenvironment<>{varblock}[2][.9\textwidth]{%
  \setlength{\textwidth}{#1}
  \begin{actionenv}#3%
    \def\insertblocktitle{#2}%
    \par%
    \usebeamertemplate{block begin}
}{\par%
    \usebeamertemplate{block end}%
  \end{actionenv}
}


\linespread{1.05} % manual set one-half spacing

\xpatchcmd{\itemize}
{\def\makelabel}
{\ifnum\@itemdepth=1\relax
    \setlength\itemsep{2ex}% separation for first level
\else
    \ifnum\@itemdepth=2\relax
    \setlength\itemsep{1ex}% separation for second level
    \else
    \ifnum\@itemdepth=3\relax
        \setlength\itemsep{0.75ex}% separation for third level
\fi\fi\fi\def\makelabel
}
{}
{}

